\documentclass[a4paper]{article}
\input{/Users/huangchengwei/Documents/LaTex Documents/LaTeX Docs/Preambles/Preamble.tex}
\theoremstyle{definition}
\newtheorem{Theorem}{Theorem}
\newtheorem{Lemma}{Lemma}
\newtheorem*{Proof}{Proof}
\newtheorem{Problem}{Problem}
\newtheorem*{Solution}{Solution}

\title{Introduction to Ordinary Differential Equations: Homework 1}
\author{C. W. Huang}
\date{2022-09-12}


\begin{document}
\maketitle
\begin{Problem}
Find the general solution of
\begin{itemize}
    \item[(a)] 
    $\dfrac{\dd^{2}y}{\dd x^{2}}-2\dfrac{\dd y}{\dd x}+y=x^{\frac{3}{2}}e^{x}$
    \item[(b)]
    $\dfrac{\dd^{2}y}{\dd x^{2}}+\dfrac{1}{x}\dfrac{\dd y}{\dd x}+\left(1-\dfrac{1}{4x^{2}}\right)y=x$
\end{itemize}
1, State your method and explain why it works.
2, Show your calculations.

\begin{Solution}[to Problem 1.]\mbox{}
\begin{itemize}
\item[(a)]
1, State your method and explain why it works:\\
Consider the associated homogeneous equation first:
\begin{align*}
y''-2y'+y=0 
\end{align*}
It is not hard, by looking at the constant coefficients, to find that $e^{x}$ is a solution to this homogeneous equation.
Next, find the other homogeneous solution by using the method of reduction of order.
After finding the homogeneous part of general solution, it remains to determine the particular integral, since we already have two linearly independent homogeneous solutions at hand, we apply the method of variation of parameters to complete the general solution.\\
2, Show your calculations:\\
First, consider the homogeneous part of the equation:
\begin{align}
    y''-2y'+y=0 \label{pr1eq0}
\end{align}
By looking at coefficients of this equation, it is clear that $u_{1}(x)=e^{x}$ is a solution to this homogeneous equation.
We apply the method of reduction of order to find another homogeneous solution. Put
\begin{align}
    y(x)&= U(x) \cdot u_{1}(x) = U\cdot e^{x}\label{pr1eq1}
    \intertext{Differentiation rules yield:}
    y' &= U'\cdot e^{x}+U\cdot e^{x} = (U'+U)e^{x}\label{pr1eq2}\\
    y'' &= (U''+U')e^{x}+(U'+U)e^{x} = (U''+2U'+U)e^{x}\label{pr1eq3}
\end{align}
Substitute the equations (\ref*{pr1eq1}), (\ref*{pr1eq2}), and (\ref*{pr1eq3}) into (\ref*{pr1eq0}), we get
\begin{align}
    (U''+2U'+U)e^{x} -2(U'+U)e^{x} +Ue^{x}=0
\end{align}
Since $e^{x}$ is never zero, we eliminate this.
\begin{align}
    (U''+2U'+U) -2(U'+U) +U=0.
\end{align}
Tidying up by collecting the terms involving $U$, $U'$ and $U''$ respectively, we get
\begin{align}
    U''=0.\label{pr1eq4}
\end{align}
Integrate (\ref*{pr1eq4}) twice, we get
\begin{align}
    U(x) = \int U'(x) \dd x = \int \left(\int U''(x)\dd x\right)\dd x = c_1x+c_2,
\end{align}
for some constants $c_{1},c_{2}$. Therefore, we have
\begin{align}
    y(x) = U\cdot e^{x}  =c_{1}xe^{x}+c_{2}e^{x},
\end{align}
and we find 
\begin{align}
    u_{2}(x)= xe^{x}
\end{align}
is another homogeneous solution.\\

So far, we have homogeneous part of the general solution, and it remains to find the particular integral. 
We complete this by using the method of variation of parameters: Put
\begin{align}
    y(x)&=c_{1}(x)u_1(x)+c_{2}(x)u_{2}(x)=c_1u_1+c_2u_2.\label{pr1eq5}
\intertext{By Differentiation}
y' &=c_1'u_1+c_1u_1'+c_2'u_2+c_2u_2'
\intertext{At this point, we impose a condition
\begin{align}c_1'u_1+c_2'u_2=0\end{align}
 and then}
y' &=c_1u_1'+c_2u_2'\label{pr1eq6}\\
y'' &= c_1'u_1'+c_1u_1''+c_2'u_2'+c_2u_2''\label{pr1eq7}
\end{align}
Now, substitute the equations (\ref*{pr1eq5}), (\ref*{pr1eq6}) and (\ref*{pr1eq7}) into original equation
\begin{align}
    y''-2y'+y=x^{\frac{3}{2}}e^{x}.
\end{align}
The left-hand side is 
\begin{align}
&(c_1'u_1'+c_1u_1''+c_2'u_2'+c_2u_2'')-2(c_1u_1'+c_2u_2')+c_1u_1+c_2u_2
\intertext{Tidying up by collecting the tems involving $c_1$, $c_1'$, $c_2$ and $c_2'$}
&=c_1(u_1''-2u_1'+u_1)+c_2(u_2''-2u_2'+u_2)+c_1'u_1'+c_2'u_2'
\intertext{Since $u_{1}$ and $u_2$ are homogeneous solutions, $u_1''-2u_1'+u_1=u_2''-2u_2'+u_2=0$}
&=c_1'u_1'+c_2'u_2'.
\end{align}
We get
\begin{align}
    c_1'u_1'+c_2'u_2' = x^{\frac{3}{2}}e^{x}\label{pr1eq8}
\end{align}
Now, we solve the system of equations (13) and (\ref*{pr1eq8}):
\begin{align}
        \left\{\begin{aligned}
        c_1'u_1+c_2'u_2&=0\\
        c_1'u_1'+c_2'u_2'&=x^{\frac{3}{2}}e^{x}
    \end{aligned}\right.\label{pr1eq9}
\end{align}
Substitute $u_{1}(x)=e^{x}$ and $u_{2}(x)=xe^{x}$ into (\ref*{pr1eq9}).
\begin{align}
\left\{\begin{aligned}
    c_1'e^x + c_2'xe^{x}&=0\\
c_1'e^x +c_2'(1+x)e^{x}&=x^{\frac{3}{2}}e^{x}
\end{aligned}\right.
\end{align}
Then we have
\begin{align}
\left\{\begin{aligned}
c_1' &=-x^{\frac{5}{2}}\\
c_2' &=x^{\frac{3}{2}}
\end{aligned}\right.
\end{align}
Integrating yields
\begin{align}
    \left\{\begin{aligned}
    c_1 &=-\frac{2}{7}x^{\frac{7}{2}}+A\\
    c_2 &=\frac{2}{5}x^{\frac{5}{2}}+B
    \end{aligned}\right.
\end{align}
where $A$ and $B$ are constant.
Therefore, the general solution is 
\begin{align}
    y(x) &=c_1u_1+c_2u_2\\
    &=\left(-\frac{2}{7}x^{\frac{7}{2}}+A\right)e^{x}+\left(\frac{2}{5}x^{\frac{5}{2}}+B\right)xe^{x}\\
    &=Ae^{x}+Bxe^{x}+\left(
        -\frac{2}{7}x^{\frac{7}{2}}+\frac{2}{5}x^{\frac{7}{2}}
    \right)e^{x}\\
    &=Ae^{x}+Bxe^{x}+\frac{4}{35}x^{\frac{7}{2}}e^{x},
\end{align}


\item[(b)]
1, State your method and explain why it works:\\
Firstly, we consider the associated homogeneous equation, and we want to find out the homogeneous solutions. 
Since the coefficients of this homogeneous equation are polynomials, we can apply the method of Frobenius. 
After finding the homogeneous solutions, we apply the method of variation of parameters to find the particular integral, complete the calculations.

2,\\
Put
\begin{align}
    y(x) &= \sum_{n=0}^{\infty} a_{n}x^{c+n}
\intertext{Then we have}
y'(x) &= \sum_{n=0}^{\infty} a_{n}(c+n)x^{c+n-1}\\
y''(x) &= \sum_{n=0}^{\infty} a_{n}(c+n)(c+n-1)x^{c+n-2}
\end{align}
The Left-hand side of the homogeneous equation
\begin{align}
    y''+\frac{1}{x}y+\left(1-\frac{1}{4x^{2}}\right)y=0
\end{align}
is
\begin{align}
    &y''+\frac{1}{x}y'+\left(1-\frac{1}{4x^{2}}\right)y\\
    &= \sum_{n=0}^{\infty} a_{n}(c+n)(c+n-1)x^{c+n-2}+\frac{1}{x}\sum_{n=0}^{\infty}a_{n} (c+n)x^{c+n-1}+\left(1-\frac{1}{4x^{2}}\right)\sum_{n=0}^{\infty} a_{n}x^{c+n}\\
    &= \sum_{n=0}^{\infty} a_{n}(c+n)(c+n-1)x^{c+n-2}+\sum_{n=0}^{\infty}a_{n} (c+n)x^{c+n-2}
    + \sum_{n=0}^{\infty} a_{n}x^{c+n}+\sum_{n=0}^{\infty} -\frac14a_{n}x^{c+n-2}
    \\ 
    &=
     \sum_{n=0}^{\infty} a_{n}(c+n)(c+n-1)x^{c+n-2}
    +\sum_{n=0}^{\infty} a_{n} (c+n)x^{c+n-2}
    +\sum_{n=2}^{\infty} a_{n-2}x^{c+n-2}
    +\sum_{n=0}^{\infty} -\frac14a_{n}x^{c+n-2}
    \\
    &=
     \sum_{n=2}^{\infty} a_{n}(c+n)(c+n-1)x^{c+n-2}
    +\sum_{n=2}^{\infty} a_{n} (c+n)x^{c+n-2}
    +\sum_{n=2}^{\infty} a_{n-2}x^{c+n-2}
    +\sum_{n=2}^{\infty} -\frac14a_{n}x^{c+n-2}
    \\
    &\quad +a_{0}c x^{c-2}+a_{1}(c+1)x^{c-1}+a_{0}c(c-1)x^{c-2}+a_{1}(c+1)c x^{c-1}-\frac{1}{4}a_{0}x^{c-2}-\frac{1}{4}a_{1}x^{c-1}
\end{align}
Since the right-hand side is zero, we have the equations:
\begin{align}
    a_{0}\left(c-\frac{1}{2}\right)\left(c+\frac{1}{2}\right)=0\\
    a_{1}\left(c+\frac{3}{2}\right)\left(c+\frac{1}{2}\right)=0
\end{align}
We choose $c=-\frac{1}{2}$ and thus $a_{0}$ and $a_{1}$ are arbitrary.

In equation (37), we have the recurrence relation:
\begin{align}
    a_{n} = \frac{-1}{n(n-1)}a_{n-2}.
\end{align}
This gives 
\begin{align}
    y(x) &=x^{-\frac{1}{2}} \sum_{n=0}^{\infty}a_{n}x^{n}=x^{-\frac{1}{2}}\left(\sum_{n=0}^{\infty} a_{2n}x^{2n}+\sum_{n=0}^{\infty}a_{2n+1}x^{2n+1}\right)\\
    &=a_{0}x^{-\frac{1}{2}}\cos(x)+a_{1}x^{-\frac{1}{2}}\sin(x)
\end{align}
Now, we apply variation of parameters tp get the particular integral:  Put
\begin{align}
    y(x)&=c_{1}(x)u_1(x)+c_{2}(x)u_{2}(x)=c_1u_1+c_2u_2.
\intertext{By Differentiation}
y' &=c_1'u_1+c_1u_1'+c_2'u_2+c_2u_2'
\intertext{At this point, we impose a condition
\begin{align}c_1'u_1+c_2'u_2=0\end{align}
 and then}
y' &=c_1u_1'+c_2u_2'\\
y'' &= c_1'u_1'+c_1u_1''+c_2'u_2'+c_2u_2''
\end{align}
Now, substitute the equations (44), (47) and (48) into original equation
\begin{align}
    y''+\dfrac{1}{x}y'+\left(1-\dfrac{1}{4x^{2}}\right)y=x.
\end{align}
One gets
\begin{align}
    c_{1}'u_{1}'+c_{2}'u_{2}'=x.
\end{align}
Now solve the system of equations (46) and (50):
\begin{align}
    \left\{\begin{aligned}
        c_{1}'u_{1}'+c_{2}'u_{2}'&=x\\
        c_{1}'u_{1}+c_{2}'u_{2}&=0
    \end{aligned}\right.
\end{align}
By Cramer's rule (provided the Wronskian is nonzero), we get
\begin{align}
    \left\{\begin{aligned}
        c_{1}' &= -x^{\frac{3}{2}}\sin x\\
        c_{2}' &= x^{\frac{3}{2}} \cos x
    \end{aligned}\right.
\end{align}
We thus have
\begin{align}
    \left\{\begin{aligned}
        c_{1} &= \int -s^{\frac{3}{2}}\sin s \dd s+A\\
        c_{2} &= \int s^{\frac{3}{2}} \cos s\dd s+B
    \end{aligned}\right.
\end{align}
and the general solution is:
\begin{align}
    y(x) &= c_{1}u_{1}+c_{2}u_{2}\\
    &= \left(\int-s^{\frac{3}{2}} \sin s\dd s+A\right)\frac{\cos(x)}{x^{\frac{1}{2}}}+\left(\int s^{\frac{3}{2}} \cos s \dd s+B\right)\frac{\sin{x}}{x^{\frac{1}{2}}}\\
    &=\underbrace{Au_{1}+B u_{2}}_{\text{homogeneous solution}}+\underbrace{\frac{\cos(x)}{x^{\frac{1}{2}}}\int-s^{\frac{3}{2}}\sin s\dd s+\frac{\sin x}{x^{\frac{1}{2}}}\int s^{\frac{3}{2}} \cos s \dd s}_{\text{particular integral}}
\end{align}




\end{itemize}
\end{Solution}

\newpage

\end{Problem}

\begin{Problem}
Find the general solution of the differential equation
$$
2x\dfrac{\dd^{2}y}{\dd x^{2}}+\left(1+x\right)\dfrac{\dd y}{\dd x}-ky=0
$$
(where $k$ is a real constant) in power series form. For which values of $k$ is there a polynomial solution.\\
1, State your method and explain why it works.
2, Show your calculations.

\begin{Solution}
    1, State your method and explain why it works:\\
Since the problem asks for solutions in power series form, we apply the method of Frobenius.\\
2, Show your calculations:\\
Put
\begin{align}
    y(x)&=\sum_{n=0}^{\infty} a_{n}x^{n+c} 
\intertext{and then}
y'(x) &=\sum_{n=0}^{\infty}a_{n}(n+c)x^{n+c-1}\\
y''(x) &=\sum_{n=0}^{\infty} a_{n}(n+c)(n+c-1) x^{n+c-2}
\end{align}
The left hand side of the equation
\begin{align}
    &2x y''+(1+x)y'-ky\\
    &= 2x\sum_{n=0}^{\infty} a_{n}(n+c)(n+c-1) x^{n+c-2}+(1+x)\sum_{n=0}^{\infty}a_{n}(n+c)x^{n+c-1}-k\sum_{n=0}^{\infty} a_{n}x^{n+c} \\
    &=\sum_{n=0}^{\infty}2 a_{n}(n+c)(n+c-1) x^{n+c-1}
    +\sum_{n=0}^{\infty}a_{n}(n+c)x^{n+c-1}
    +\sum_{n=0}^{\infty}a_{n}(n+c)x^{n+c}
    +\sum_{n=0}^{\infty} (-k)a_{n}x^{n+c} \\
    &=\sum_{n=0}^{\infty} a_{n}(n+c)(2n+2c-1)x^{n+c-1}+\sum_{n=0}^{\infty} (n+c-k)a_{n}x^{n+c}\\
    &= a_{0} c (2c-1) x^{c-1} +\sum_{n=1}^{\infty} a_{n}(n+c)(2n+2c-1)x^{n+c-1}+\sum_{n=0}^{\infty} (n+c-k)a_{n}x^{n+c}\\
    &= a_{0} c (2c-1) x^{c-1} +\sum_{n=0}^{\infty} a_{n+1}(n+1+c)(2n+2c+1)x^{n+c}+\sum_{n=0}^{\infty} (n+c-k)a_{n}x^{n+c}\\
    &= a_{0} c (2c-1) x^{c-1}+\sum_{n=0}^{\infty} \left(
        a_{n+1}(n+1+c)(2n+2c+1)+a_{n} (n+c-k)
    \right) x^{n+c}
\end{align}
Now we have the indicial equation:
\begin{align}
    c(2c-1)=0 \implies c=0 \text{ or }c=\frac{1}{2}
\end{align}
Plug in $c=0$ first: the recurrence relation is 
\begin{align}
    a_{n+1} (n+1)(2n+1)+a_{n}{(n-k)}=0 & \text{ for }n\ge 1
\end{align}
From (68), we get 
\begin{align}
    a_{n+1} = \frac{k-n}{(n+1)(2n+1)}a_{n}
\end{align}
This gives a solution
\begin{align}
    y_{1}(x) = \sum_{n=0}^{\infty} \frac{\prod_{j=1}^{n}(k+1-j)}{(n!)(2n-1)!}x^{n}
\end{align}
Now plug in $c=\frac{1}{2}$, the recurrence relation is 
\begin{align}
    a_{n+1}\left(n+\frac{3}{2}\right)\left(2n+2\right)+a_{n}\left(n+\frac12-k\right)=0
\end{align}
From (71) we get
\begin{align}
    a_{n+1} = \frac{2k-2n-1}{(2n+3)(2n+2)}a_{n} 
\end{align}
This gives a solution
\begin{align}
    y_{2}(x) = x^{\frac{1}{2}}\sum_{n=0}^{\infty} \frac{\prod_{j=1}^{n} (2k-2j+1)}{(2n+1)!}x^{n}
\end{align}
So far, we get two power series solutions $y_1$ and $y_2$. 
From the expression of them, we note that if $k$ is a nonnegative integer, then the coefficients in $y_{1}$ becomes zero from the $(k+1)$th term, and $y_{1}$ becomes a polynomial solution.
Similarly, if $k$ is of the form $\dfrac{\text{positive odd integer } m}{2}$, then the coefficients in $y_{2}$ becomes zero from the $\left(\dfrac{m+1}{2}\right)$th term, and $y_2$ becomes a polynomial solution.

We conclude that if $k=\frac{h}{2}$, where $h$ is nonnegative, then the equation has a polynomial solution.


\end{Solution}


\end{Problem}
\end{document}